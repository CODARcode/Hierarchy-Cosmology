\begin{abstract}
Lossy compression of local minimizers to a numerical optimization problem can result 
in significant savings by utilizing extra knowledge available for numerical 
optimization problems.  In particular, we tradeoff storage for computation.
If we can compute the basin of local convergence for a particular numerical
method and approximate local minimizer, then as long as the iterate we store 
remains within that basin, we can recover the approximate local minimizer.  
Moreover, we may be able to bound the computational cost of the recovery by 
establishing a bound on the number of iterations of the local method
required.  Our initial investigation will focus on unconstrained and
bound-constrained numerical optimization methods and will start with
a literature search on perturbation analysis and error bounds.

We expect that these methods will be beneficial to the Exaalt project, which needs 
to store all the local minimizers computed in a long molecular dynamics simulation.  
The number of local minimizers is exponential in the number of atoms and they are 
targeting problems with 2,000 atoms and each local minimizer needs to store 
10,000 values.  While each individual minimizer is a small number values, 
we need to conserve space as there is an enormous number of them.  Beyond 
storing the values, they are interested in clusters of local minimizers and 
the transition between such clusters; transitions between clusters of 
clusters may be of interest as well.  Additional analysis and methods
to compute transition states and minimal energy pathways may also be
of interest.  Our initial investigations will use simple systems
and/or data previously generated and stored.

Provided that we can identify the mathematics required to compute the 
basins of local convergence for the local numerical optimization methods 
employed by the Exaalt team, we will produce prototype implementations
and study their scalability and work toward reducing the computational
costs as much as possible, perhaps using hierarchical methods.  We will 
then evaluate the tradeoffs between computation of the basins, storage 
of the results, and recomputation of the local minimizers.  We will 
also evaluate the downstream cluster analysis to determine if we
need to recompute the local minimizers or if the methods are
stable when using the approximate local minimizers.  If a good 
tradeoff can be made and the methods are scalable, we will move 
the project into the next step of developing an online capability.

\end{abstract}

